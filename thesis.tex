%\documentclass[12pt,oneside]{fithesis2}		% pc version
\documentclass[a4paper, 12pt, twoside]{fithesis2}		% print version

% ===== LOADING PACKAGES =====
% language settings, main documnet language last
\usepackage[english]{babel}
% enabling new fonts support (nicer)
\usepackage{lmodern}
% setting input encoding
\usepackage[utf8]{inputenc}
% setting output encoding
\usepackage[T1]{fontenc}
% fithesis2 requires csquotes
\usepackage{csquotes}
% set page margins
%\usepackage[top=3.0cm, bottom=3.5cm, left=2.4cm, right=2.4cm]{geometry}	% pc version
\usepackage[top=3.0cm, bottom=3.5cm, left=2.9cm, right=1.9cm]{geometry}	% print version
% package to make bullet list nicer
\usepackage{enumitem}
% math symbols and environments
\usepackage{mathtools}
\usepackage{amsmath}
% packages for complex tables
%\usepackage{tabularx}
%\usepackage{multirow}
%\usepackage{dcolumn}
%\usepackage{array}
% package for defining new floating environments
\usepackage{float}
\usepackage[labelfont=]{caption}
%\usepackage{newfloat}

% bibliography management
\usepackage[backend=biber, 		% use biber as backend instead of BiBTeX
        %dashed=false                    % when there is an author twice, still write his/her name
	bibstyle=ieee-alphabetic, 	% bibliography style: IEEE with alphabetic citations
	citestyle=alphabetic, 		% citation style
	url=true, 			% display urls in bibliography
	hyperref=auto,			% detect hyperref and create links
	%block=ragged, 			% format bibliography into blocks, ragged on right
]{biblatex}
\addbibresource{thesis.bib}

% setting custom colors for links
\usepackage{xcolor}
\definecolor{dark-red}{rgb}{0.6,0.15,0.15}
\definecolor{dark-green}{rgb}{0.15,0.4,0.15}
\definecolor{medium-blue}{rgb}{0,0,0.5}
% generating hyperlinks in document
\usepackage{url}
\usepackage[plainpages=false, 	    % get the page numbering correctly
            pdfpagelabels, 	    % write arabic labels to all pages
            unicode,	 	    % allow unicode characters in links
            colorlinks=true, 	    % use colored links instead of boxed (pc version)
            %hidelinks, 		    % hide links (print version)
            linkcolor={dark-red},
            citecolor={dark-green},
            urlcolor={medium-blue}
			]{hyperref}

% ===== FI THESIS SETTINGS =====

\thesistitle{Automatic question generation\\and adaptive practice}
\thesissubtitle{Bachelor thesis}
\thesisstudent{Tomáš Effenberger}
\thesiswoman{false}
\thesisfaculty{fi}
\thesisyear{spring 2015}
\thesisadvisor{RNDr.\ Jan Rygl}
\thesislang{en}

% ===== LATEX DOCUMENT SETTINGS =====

% adjusting hyphenation penalties
\tolerance=10000
\hyphenpenalty=500

% renew command for shorter and nicer underscore
\renewcommand{\_}{\leavevmode \kern0.0em\vbox{\hrule width0.4em}}

% vertical space between paragrahps, no indentation
\usepackage[parfill]{parskip}  % TODO: nefunguje -> zfunkcnit
% space between paragraphs
%\setlength{\parskip}{0.6em plus0.2em minus0.2em}

% ===== COMMANDS =====

%--------------------------------------------------------------------
% define square symbol
%--------------------------------------------------------------------
\newcommand{\squarebullet}{\textcolor{black}{\raisebox{0.15em}{\rule{4pt}{4pt}}}}

%--------------------------------------------------------------------
% define new itemize environment with squares and smaller spaces
%--------------------------------------------------------------------
\newenvironment{myItemize}{
  \begin{itemize}[leftmargin=2em,rightmargin=1em,itemsep=\parskip ,parsep=0em,topsep=0em,partopsep=0em]
  \renewcommand{\labelitemi}{\squarebullet}
  \renewcommand{\labelitemii}{$\diamond$}
}{
  \end{itemize}
}

%--------------------------------------------------------------------
% exercise environment
%--------------------------------------------------------------------
\newcounter{choice}
\renewcommand\thechoice{\Alph{choice}}
\newcommand\choicelabel{\thechoice.}

\newenvironment{choices}%
  {\vspace{0.2em}\list{\choicelabel}%
     {\usecounter{choice}\def\makelabel##1{\hss\llap{##1}}%
       \settowidth{\leftmargin}{W.\hskip\labelsep\hskip 0.01em}%
       \def\choice{%
         \item
       } % choice
       \labelwidth\leftmargin\advance\labelwidth-\labelsep
       \topsep=0pt
       \partopsep=0pt
     }%
  }%
  {\vspace{-0.7em}\endlist}

%\setlength{\abovecaptionskip}{25pt plus 3pt minus 2pt}
\floatstyle{boxed}
\newfloat{exercise}{thp}{exrcs}[chapter]
\floatname{exercise}{Exercise}

\newenvironment{question}
{
  \begin{center}
  \begin{tabular}{p{0.9\textwidth}}
  \vskip 0.05em
}
{
  \\
  \end{tabular}
  \end{center}
}

% gap in the sentence (bottom line)
\newcommand{\sentenceGap}{\rule{1.5cm}{0.4pt}~}

%--------------------------------------------------------------------

% ===== CHEAT SHEET =====

% == full width image

%\begin{figure}[b!]
%\centering
%\includegraphics[width=\textwidth]{images/bla.bla}
%\caption{bla bla bla}
%\label{fig:bla-bla}
%\end{figure}


% == myItemize

%\begin{myItemize}
%\item \textbf{Bla}\\
%  bla
%\item \textbf{Bla}\\
%  bla
%\item \textbf{Bla}\\
%  bla
%\end{myItemize}


% ===== BEGIN DOCUMENT =====
\begin{document}

\FrontMatter
\ThesisTitlePage

\begin{ThesisDeclaration}
\DeclarationText
\AdvisorName
\end{ThesisDeclaration}

\begin{ThesisThanks}

  TODO (Sir, Papi, ALG, NLP)
%I'd like to thank Petr for his guidance, enthusiasm and inspiring discussions.
%I also owe much to my mom and brother for their continuous support. Thank you.

%\noindent
%Further thanks goes to all my friends who had to put up with my enthusiasm
%and numerous research details they may have never asked for $\ddot\smile$.

%\noindent
%Last but not least, I'd like to acknowledge the Laboratory of Security and Applied Cryptography and
%the National Grid Infrastructure MetaCentrum for providing access to their computing and storage facilities.
\end{ThesisThanks}

\begin{ThesisAbstract}
When studying, it is more efficient not just to read about the topic, but also to practice the knowledge, e.g. by answering some multiple choice questions. Today, there is a huge amount of information to study (consider Wikipedia) and it is not possible to create a set of question for all topics manually. However, we can generate questions automatically, using techniques of artificial intelligence and natural language processing. This thesis explores the state-of-the-art approaches to question generation and describes their advantages and disadvantages. The thesis also suggests a design of the general framework for practicing knowledge from articles. This framework is implemented and publicly accessible through a web interface.
\end{ThesisAbstract}

\begin{ThesisKeyWords}
knowledge representation, question generation, adaptive practice, learning,
artificial intelligence, natural language processing
\end{ThesisKeyWords}

\MainMatter
\tableofcontents

% ===========================  CHAPTER ===========================
\chapter{Introduction}
\label{chap:intro}

TODO: intro, goal of the thesis (dodelat podle oficialniho zadani / abstraktu)

Mere repeated reading of the text which one is traying to learn is unefficient method of learning, while answering to related questions leads to [better long-term knowledge / dlouhodobemu zapamatovani] \parencite{edu-improve}.

In spite of active and [dlouhodoby nebo tak neco] research of question generation (e.g. \parencite{questions-wolfe, questions-eval}) [mozna pridat dalsi, pripadne rozepsat - priklad early research a recent research],
there is still no publicly available web application to solve this task.
And I am sure that such application would be really useful. Typical example is to make self-studying more efficient by answering a few generated questions after reading an article to verify and consolidate the new knowledge.
[overit aktualnost a jasne urcit moment o kterem mluvim (neexistuje tady a ted)]

That is why I have implemented \textit{Smartoo, Smart Artificially Intelligent Tutor}, modular and extensible framework for question generation and adaptive practice
of knowledge from Wikipedia [taky citovat!] articles.
The Smartoo Framework is licensed under the GNU General Public License, version 2 [citovat?, licenci rozmyslet].
The source code is available from project's page on \textit{GitHub} \parencite{smartoo-github}.
I have used the framework to create a simple [instance/demo/prototype?] of the web application for practicing knowledge from Wikipedia articles \parencite{smartoo-web}.

The whole practicing process consists of four steps.
First step is to extract facts from the given article.
Besides the knowledge extraction itself, knowledge representation is an important issue as well.
Extraction and representation of the knowledge will be the topic of chapter \ref{chap:knowledge}.

In chapter \ref{chap:exercises}, excercises generation will be discussed.
Usually, exercises will be just multiple choice questions, but other exercises types are possible as well.
In this step major concerns are: selection of a fact (or set of facts) from which the exercise will be generated, transformation from the fact (or facts) to the exercise and in case of multiple choice questions selection of good \textit{distractors} (false answers).

Although we could stop here and submit all generated questions to the user, it has been proven to be useful to make two more steps.
In chapter \ref{chap:exercises-grading}, I will talk about exercises grading.
We can be interested in various parameters, such as difficulty, relevance to the article or probability that the question is gramatically [?].

Having the exercises graded, we can filter them and only present these with reasonable difficulty, relevance and correctness probability.
But again, we can introduce one more step to make learning more efficient -- adaptive practice.
According to the user performance, we can choose easier or more difficult questions. I will talk about some common strategies for this task in chapter \ref{chap:practice}.

In chapter \ref{chap:smartoo}, I will describe the Smartoo Framework in detail
and we will see how these four steps are transparently glued and how a component for each step should look like.
I will also mention how each of the four components is implemented in the online prototype system.

Smartoo is designed to collect both implicit and explicit feedback from users.
I will analyze feedback from about one month testing period in chapter \ref{chap:evaluation}.

Finally, in the chapter \ref{chap:future}, I will present planned future development of Smartoo application.


%The thesis text was typeset in \LaTeX{} using the \textit{fithesis2} package created by Stanislav Filipčík \parencite{fithesis}.

%TODO: (?) The text of the thesis is licensed under a Creative Commons Attribution 3.0 Unported License.

% ===========================  CHAPTER ===========================
\chapter{Knowledge extraction}
\label{chap:knowledge}

TODO...

Knowledge extraction is well-studied aria \parencite{triples-acquisition}.

\section{Knowledge bases}
\label{sec:knowledge-bases}

TODO...
Besides the text itself external knowledge bases can be used.
Examples of popular knowledge bases are DBpedia \parencite{dbpedia} and Freebase [citovat!].
DBpedia was built by extracting structured information from Wikipedia.
Facts are stored in the form of \textit{RDF} (\textit{Resource Description Framework}) graph [citovat],
which is a collection of subject -- predicate -- object triples \parencite[][63]{semantic-web}.

RDF graph is often used for knowledge representation.
This format is suitable for exercise genereation, as there is a tool for convenient querying RDF graph,
\textit{SPARQL}%
\footnote{The name of this query language is a recursive acronym of \textit{SPARQL Protocol and RDF Query Language}.}
language \parencite[][84]{semantic-web}.



% ===========================  CHAPTER ===========================
\chapter{Exercises generation}
\label{chap:exercises}

TODO...

\section{Distractors}
\label{sec:distractors}

It is important to present a multiple-choice question with competitive \textit{distractors} (incorrect alternatives).
As shown by experiments described in \cite{optimizing-multiple-choice}, multiple-choice questions with competitive incorrect alternatives not only help to remember the correct answers to the original questions, but they also improve later performance on the previously nontested questions for which these alternatives may be the correct answers. If the alternatives are not competive, a student can recognize the correct answer without retrieval, just by pattern matching. As an example, consider the following multiple-choice question:

\begin{exercise}
\caption{Question with noncompetitive alternatives}%\label{table:somename}
  \begin{question}
  Lincoln was assassinated by \sentenceGap , a Confederate sympathizer.
  \begin{choices}
    \choice Emancipation Proclamation
    \choice John Wilkes Booth
    \choice Illinois
    \choice Department of Agriculture
  \end{choices}
  \end{question}
\end{exercise}

Clearly, the student does not need to remember who assasinated Lincoln to answer this question correctly and they are also not forced to retrieve any knowledge about alternatives.
Of course, this was an extreme example to illustrate the point -- you definitely would not come accross such poor distractors in manually created multiple-choice questions.
Now take a look at the following question:

\begin{exercise}
\caption{Question with competitive alternatives}%\label{table:somename}
  \begin{question}
  Lincoln was assassinated by \sentenceGap , a Confederate sympathizer.
  \begin{choices}
    \choice Thomas N. Conrad
    \choice Robert E. Lee
    \choice John Wilkes Booth
    \choice Ward Hill Lamon
  \end{choices}
  \end{question}
\end{exercise}

This questions is much more likely to force student to retrieve knowledge about John Wilkes Booth as well as about other alternatives.

\cite{optimizing-multiple-choice} also showed that competitive distractors are not confused with correct answers in later questions more often than noncompetitive distractors.

The consequences are clear -- when creating a multiple-choice question, we should try to find plausible alternatives which are as competitive as possible.




% ===========================  CHAPTER ===========================
\chapter{Exercises grading}
\label{chap:exercises-grading}

NOTE: mozna neni na samostatnou kapitolu (v tom pripade spojit s predchozi, ale je to tak vice konzistentni

% ===========================  CHAPTER ===========================
\chapter{Adaptive practice}
\label{chap:practice}

TODO...

% ===========================  CHAPTER ===========================
\chapter{Smartoo framework}
\label{chap:smartoo}

TODO...

podkapitoly pro jednotlive komponenty

\section{Knowledge building}
\label{sec:smartoo-knowledge}

TODO...

\section{Exercises creating}
\label{sec:smartoo-exercises}

TODO...

\section{Exercises grading}
\label{sec:smartoo-exercises-grading}

TODO...

\section{Adaptive practice}
\label{sec:smartoo-practice}


TODO...

\section{Web interface}
\label{sec:smartoo-web}


TODO...

% ===========================  CHAPTER ===========================
\chapter{Evaluation}
\label{chap:evaluation}

TODO...

performace of each component

\chapter{Future plans}
\label{chap:future}

TODO...

what to improve (... in diploma thesis)





% ===== APPENDIX AND BIBLIOGRAPHY =====
\appendix

% include citations not cited specifically
%\nocite{*}

% print complete bibliography
\printbibliography

\chapter{Data attachment}

TODO: data attachment structure description

\end{document}
